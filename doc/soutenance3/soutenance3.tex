%%%%%%%%%%%%%%%%%%%%%% Props %%%%%%%%%%%%%%%%%%%%%%
\documentclass{article}

\usepackage[french]{babel}
\usepackage[utf8]{inputenc}
\usepackage[T1]{fontenc}
\usepackage{graphicx}
\usepackage{fancyhdr}
\usepackage{eurosym}
\usepackage{color}
\usepackage{soul}
\usepackage{listings}
\usepackage{enumitem}
\usepackage{enumerate}

\pagestyle{fancy}
\lhead{Soutenance 3}
\chead{Deadly Science}
\rhead{Custos Carceris}

\definecolor{mygreen}{rgb}{0,0.6,0}
\definecolor{mygray}{rgb}{0.5,0.5,0.5}
\definecolor{mymauve}{rgb}{0.58,0,0.82}

\lstset{ 
  commentstyle=\color{mygreen},
  keywordstyle=\color{blue},       % keyword style
  numberstyle=\tiny\color{mygray}, % the style that is used for the line-numbers
  rulecolor=\color{black},         % if not set, the frame-color may be changed on line-breaks within not-black text (e.g. comments (green here))
  stringstyle=\color{mymauve},     % string literal style
  language=[Sharp]C,                 % the language of the code
  backgroundcolor=\color{white},   % choose the background color; you must add \usepackage{color} or \usepackage{xcolor}; should come as last argument
  basicstyle=\footnotesize,        % the size of the fonts that are used for the code
  breakatwhitespace=true,         % sets if automatic breaks should only happen at whitespace
  breaklines=true,
  extendedchars=true,              % lets you use non-ASCII characters; for 8-bits encodings only, does not work with UTF-8
  frame=single,	                   % adds a frame around the code
  tabsize=2,	                   % sets default tabsize to 2 spaces
  showstringspaces=false,
  numbers=left,
}

\begin{document}


%%%%%%%%%%%%%%%%%%%%%% Titre %%%%%%%%%%%%%%%%%%%%%%
\begin{titlepage}
	\centering
	{\scshape\LARGE Custos Carceris\par}
	\vspace{1cm}
	{\scshape\Large Soutenance 3 \par}
	\vspace{1.5cm}
	{\huge\bfseries Deadly Science\par}
	\vspace{2cm}
	\includegraphics[width=0.5\textwidth]{logo.png}\par\vspace{1cm}
	{\Large\itshape Léandre Perrot\par}
	{\Large\itshape Yann Boudry\par}
	{\Large\itshape Steve Suissa\par}
	{\Large\itshape Célian Raimbault\par}
	\vfill
	Un projet EPITA
	\vfill
	{\large \today\par}
\end{titlepage}



\newpage
\tableofcontents


%%%%%%%%%%%%%%%%%%%%% Intro %%%%%%%%%%%%%%%%%%%%%%

\newpage
\section{Introduction}

%%%%%%%%%%%%%%% TODO : Mettre a jour
Le jeu a grandement avancé depuis le retour du cahier des charges : le réseau est quasiment entièrement fonctionnel, pareil pour la caméra, le site avance petit à petit, la génération est presque fini et certaines musiques ont été composées. Globalement, le jeu est quasiment jouable.
\begin{table}[!h]
\centering
\caption{Avancement}
\begin{tabular}{|l|l|l|}

\hline
%%%%%%%%%%%%%%% TODO : Mettre a jour
Tâches $\backslash$Soutenances & Attendu & Réalité \\ \hline
Camera & 50\% & 75\% \\ \hline
G Joueur & 30\% & 50\% \\ \hline
G Jeu & 30\% & 30\% \\ \hline
Reseau & 50\% & 75\% \\ \hline
Map Const & 30\% & 30\% \\ \hline
Menu & 15\% & 70\% \\ \hline
Chrono/GUI & 30\% & 30\% \\ \hline
Site TXT & 0\% & 30\% \\ \hline
Site E & 0\% & 0\% \\ \hline
Map Gen & 0\% & 50\% \\ \hline
Musique & 0\% & 25\% \\ \hline
Sons & 0\% & 10\% \\ \hline

\end{tabular}
\end{table}
 
\newpage
\section{Réalisations}


%%%%%%%%%%%%%%%%%%%%%% Yann %%%%%%%%%%%%%%%%%%%%%%%%%%%%%%
\subsection{Yann}

\subsubsection{Menu principal}
%%%%%%%%%%%%%%%%%% TODO : Changer
\paragraph{Accueil}
Comme vous pouvez le voir il n'a pas évolué pour l'instant mais des images ou video viendront le garnir dans sa version finale, lorsqu'il n'y aura plus de modification trop importantes.

%%%%%%%%%%%%%%%%%%%%%% Celian %%%%%%%%%%%%%%%%%%%%%%%%%%%%%%
\newpage
\subsection{Célian}


%%%%%%%%%%%%%%%%%%%%%% Léandre %%%%%%%%%%%%%%%%%%%%%%%%%%%%%%
\newpage
\subsection{Léandre}


%%%%%%%%%%%%%%%%%%%%%% Steve%%%%%%%%%%%%%%%%%%%%%%%%%%%%%%

\newpage
\subsection{Steve}




\newpage
\section{Conclusion}

%%%%%%%%%%%%%%%%%%%%%%%%%%%%%% TODO : Mettre a jour
En ce qui concerne la prochaine soutenance, le jeu devrait être à peu de choses près fini. D'ici là, il nous reste encore à paufiner les décors et les musiques, à créer les "salles d'attente" pour la génération des joueurs, améliorer la seconde phase, compléter le site en y incluant également le lien de téléchargement du jeu, rendre les menus plus fluides et ergonomiques, et également mettre au point un système permettant d'installer le jeu et de le désinstaller. Autrement dit, il nous reste encore du pain sur la planche, mais ce devrait être largement faisable dans les délais imposés pour le Projet S2.

\emph{If you knew time as well as I do, you would play Deadly Science.}


\end{document}



